% Created 2020-10-29 Thu 14:24
% Intended LaTeX compiler: pdflatex
\documentclass[11pt]{article}
\usepackage[utf8]{inputenc}
\usepackage[T1]{fontenc}
\usepackage{graphicx}
\usepackage{grffile}
\usepackage{longtable}
\usepackage{wrapfig}
\usepackage{rotating}
\usepackage[normalem]{ulem}
\usepackage{amsmath}
\usepackage{textcomp}
\usepackage{amssymb}
\usepackage{capt-of}
\usepackage{hyperref}
\author{Exr0n}
\date{\today}
\title{Lin Alg flo 19}
\hypersetup{
 pdfauthor={Exr0n},
 pdftitle={Lin Alg flo 19},
 pdfkeywords={},
 pdfsubject={},
 pdfcreator={Emacs 27.1 (Org mode 9.3)},
 pdflang={English}}
\begin{document}

\maketitle
\tableofcontents

\section{Broader vector spaces}
\label{sec:org292b16e}
\begin{itemize}
\item Doesn't have to be physics vectors
\item maybe it's like matrices
\item or linear maps themselves
\end{itemize}
\section{The Linear Map 0}
\label{sec:org1fbf993}
A linear map \(S = 0\) is a map where \(Su = 0 \forall u\).
\section{Axler 3.A ex7}
\label{sec:org3c3404f}
Let \(w = Tv\).

\subsection{If \(v = 0\) then}
\label{sec:orgcf7939a}
$$Tv = 0$$
By Axler 3.11 (Maps take 0 to 0). Thus, \(\lambda\) can be anything in \(\mathbb F\).

\subsection{Otherwise,}
\label{sec:org95a76ea}
\(\frac{1}{v} \in \mathbb F\) because the field has multiplicative inverses for all elements except 0.
$$
   Tv = w = \left( w \frac{1}{v} \right)v
   $$
Let \(\lambda = w \frac{1}{v}\), then
$$ \lambda v = w \frac{1}{v} v = w $$
which is in \(\mathbb F\) because \(w, \frac{1}{v} \in \mathbb F\) and fields are closed under multiplication.

\section{Axler 3.A ex10}
\label{sec:orgb4d7399}
The additivity of a linear map \(T\) requires \(T(u+v) = Tu + Tv\). Because \(U \subset V, U \neq V\), there must be some element \(v \in V\) yet \(v \notin U\).

For some element \(u \in U\),
$$Tu + Tv = Su + 0 = Su$$
Yet \(u+v \notin U\) because if it were, then \((u+v)+(-v) = v\) would be in \(U\). Thus,
$$T(u+v) = 0$$

Because for some \(u\) \(Su\neq 0\), additivity does not hold over \(T\) and thus the map is not linear.
\end{document}
